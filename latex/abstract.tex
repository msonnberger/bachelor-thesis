In order to increase passenger numbers and customer satisfaction in public transport networks, transit authorities have to take various measures to provide an attractive alternative to other forms of transport. Besides offering an extensive network and improving infrastructure, one of those measures is reducing delays and providing reliable services. To efficiently reduce delays, it is desirable to detect hotspots in the network where delays occur the most. This thesis aims to analyze departure data to find potential hotspots and answer whether those hotspots are at transfer hubs and terminal stations. To do so, departure data in Vienna's public transport network was collected over 30 days. After reviewing previous research conducted in other cities, the results of the analysis and possible reasons for the results will be discussed in this thesis. Additionally, visualizations of the collected delay data will be presented to the reader in order to show the results in a more easy-to-understand way.