\section{Discussion}
\label{sec:discussion}

In this section, we will explore and discuss possible reasons for the results found in \cref{sec:analysis} and \cref{sec:visualization}.

\subsection{Line and station overview}

In \cref{sec:analysis}, we at first showed the most and least delayed lines in the network. While many lines had a perfect punctuality rate of 100\%, most of those lines are either really short or have special routes where they face very little traffic. Line 2A for example is a line that has 10 stops within the city center where traffic is rather low due to the higher number of pedestrian areas. Line ZF is a special line that travels around the Central Cemetery and thus is not affected by any traffic at all. Lines in \cref{table:overall-top10-most} that start with the letter \enquote{N} are night lines, which again face very little traffic during the night. Because of those reasons, it is not surprising that these lines have punctuality rates of 100\%. Lines 34A and 73A are also comparatively short lines, having 19 and 16 stations respectively. Additionally, many trips on those routes do not run all the way to the last station but rather end at an earlier one, resulting in trips with 10--15 stations, significantly less than the average bus route. The lines with the lowest punctuality rates all have relatively similar mean delays, ranging from 24.87 to 51.89 seconds. The least punctual line 42A, however, has a significantly higher mean delay of 149.01 seconds. Since this observation is also visible in one of the visualizations, it will be further discussed in \cref{discussion:vis}.

When looking only at metro lines, it became visible that all five lines have very high punctuality rates with the lowest value at 98.65\%. Since metro lines are entirely separated from both external traffic and each other (i.e. the individual lines do not share any tracks), those high percentages are within expectations. Excluding U6E, which is a special replacement line during construction work, all metro lines showed similar mean delays of around 10 seconds, which indicates that there are no outliers, at least when looking at the lines as a whole.

Tram lines on the other hand showed more spread-out results, which seems obvious when considering that traffic can have great effects on how smoothly a tram can operate through narrow city streets. Interestingly, the data shows that the most punctual lines are lines with fewer stops than the least punctual lines which are all relatively long lines. The five most punctual tram lines have an average of 15.8 stops, while on the other hand, the five least punctual lines have an average of 31.8 stops, offering twice as many opportunities for delayed departures and therefore being a plausible cause for those higher delay rates. An interesting observation is that three out of the five most punctual tram lines have negative mean delays. This effect is likely caused by the terminal stations, where trams have scheduled dwell times of 5--10 minutes so that the next trip can start punctually and drivers have the opportunity for a small break. Even though the negative values only occur at the terminal stations, the mean delay of the entire line is also negative. This again is a consequence of the short length of those lines. Since this especially becomes visible when looking at each stop of one line, it will be further discussed in the next section.

Next, all departures were grouped by stations and their punctuality rates were aggregated. The results showed that there is one station with significantly more delays than others. While the station in question, \textit{Blaasstraße}, is part of the second most delayed line 10A, at first sight, it is not affected by any special circumstances which could explain the drastically lower punctuality rate. This single outlier is also visible when looking at the box plots in \cref{fig:boxplot-stations}. Even when comparing the station to its direct neighbors on the line, namely \textit{Dänenstraße} and \textit{Hardtgasse}, the anomaly cannot be explained since those two stations show delay distributions in line with the overall trend. \textit{Blaasstraße} however even has a median delay of 60 seconds, whereas one would expect the median as well as the \nth{75} percentile to be 0, based on the rest of the data. To further confirm that there is no external reason for these high numbers, a local inspection and test rides on line 10A were carried out. Since this did not result in any new findings, a scheduling error or another non-obvious reason seems likely.

Looking only at stations that serve metro lines, one interesting observation is that out of the five stations with the most delays, shown in \cref{table:metro-stations-top5-least}, three have metro, tram and bus services and thus lots of transfers occurring. The five most punctual stations on the other hand, shown in \cref{table:metro-stations-top5-most}, all only serve exactly one metro line and buses, but no trams, which makes them less susceptible to delays happening due to large crowds interchanging between lines.
Overall though, all of those stations have high punctuality rates, with the difference between the most and least delayed stations only being 2.8 percentage points, which the box plot in \cref{fig:boxplot-stations} clearly demonstrates as well.

\subsection{Tram line 71 and station \textit{Karlsplatz}}

After analyzing the collected data on a more general level by looking at all lines or stations at once, \cref{sec:line-71-analysis} focused on one particular line, namely tram line 71. When looking at both punctuality rates and mean delays, one can observe that for lower punctuality rates, mean delay increases. However, for some stations such as \textit{Ring/Volkstheater U} or \textit{Oper/Karlsplatz U} this pattern does not hold, meaning that these stations have a higher mean delay than the respective next ones in the ordered list. A plausible explanation for this is that both these stations are important interchange stations for metro lines (indicated by the letter \enquote{U} in the name). At such interchanges, it is more likely that delays occur due to big crowds entering and exiting the vehicles and thus increasing dwell time. The fact that these inflated numbers do not necessarily carry on to the next stations indicates that the network operator Wiener Linien has included this fact in their schedule planning. Regarding punctuality rates, it becomes obvious that delays gathered in the city center do accumulate to the end of the line. The most punctual stations at the top of \cref{table:tram-71-stations} are all within the first few stops in the center while stations with low punctuality rates are at the end of line 71. One eye-catching exception is the last station but very first entry, \textit{Kaiserebersdorf, Zinnergasse}. While all other stations have relatively normal mean delays of 18 to 30 seconds, this station has a mean delay of -25.15 seconds. Trams in Vienna usually have a 5--10-minute stay at the terminal stations, enabling a short break for the driver and, as importantly, making up for accumulated delays and ensuring a punctual departure on the trip back. This explains the excellent punctuality rate for this particular station and likely other terminal stops as well. Since early departures are generally not desired though, there have to be more causes for this effect of negative departure delays. One of those possible causes could be that consistent intervals are preferred to perfectly following the schedule, resulting in drivers departing early for their next trip instead of waiting, which itself could cause an accumulation of waiting vehicles at the terminal station.

In addition to one single line, one station was analyzed in a similar manner, shown in \cref{table:karlsplatz-lines}. The station of choice was \textit{Karlsplatz}, as it is one of the busiest stations with metro, trams and bus lines stopping there. Metro lines again showed the highest punctuality rates and lowest mean delays. This again is most likely an effect of metros being entirely separated from other traffic, unlike trams and buses. The negative mean delay of -16.2 seconds for U2Z shows the same effect as the terminal stop of line 71 mentioned above. U2Z terminates at \textit{Karlsplatz}, meaning that it has enough leeway scheduled to make up for possible delays. In general, though, trams show worse punctuality rates and mean delays than buses at \textit{Karlsplatz}. While tram lines 71, D, 62 and 2 show consistent mean delays of around 30 seconds, line 1 has a significantly lower value with just 0.76 seconds. One possible explanation for this is that schedule planning is slightly more accurate for this line compared to the others mentioned.


\subsection{Visualizations}
\label{discussion:vis}

In \cref{sec:visualization}, two types of visualizations were presented which allow for an intuitive understanding of the large amounts of data at hand. The first one showed a map of Vienna with all stations colored on a gradient from red to green, respective to their punctuality rate. A sequence of bus stops on line 42A was visibly more yellow than the rest of the network. Line 42A is a rather short line with twelve stops, none of which are located within the inner districts, where more traffic and passenger volume would be expected. While previous delays do accumulate towards the end of a line as seen in previous sections, it is interesting to see the effect this strongly at this line and leaves the question if there exist other reasons for this behavior.

The same visualization but with only metro and tram stations shown discovered that the stations \textit{Gredlerstraße} and \textit{Marsonogasse} were the two most delayed tram stations in the network. While the former is a regular station part of line 2 (one of the most delayed trams), the latter is a special station located at one of the tram depots, where vehicles only stop at the very beginning and end of service. In \cref{fig:metro-only-stations}, where only metro stations are shown, it became again apparent that delays that accumulate through the trip of a metro line are accounted for at the terminal stop by calculating time reserves accordingly. This pattern is visible at most terminal stops, where their immediate neighbors show a considerably lighter shade of green or even reaching to the end of the color scale like for the U2 station \textit{Aspern Nord}, located in the far east of Vienna.

For the next visualization, all departures were grouped by the hour of the day and then plotted onto a line graph with both the mean delay and the number of departures visible. In order to see the potential effect of rush hour traffic more clearly, the departures were split into weekdays and weekends and visualized separately. This effect was indeed clearly visible, especially on weekdays when the punctuality rate was significantly lower at \DTMtime{08:00:00} and \DTMtime{16:00:00}. On Saturdays and Sundays however, the image was very different, not showing any rush hours but with minimums at \DTMtime{02:00:00} and \DTMtime{15:00:00}.