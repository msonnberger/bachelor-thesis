\subsection{Data availability}

\subsubsection{\acf{HAFAS}}

In Germany, Austria and Switzerland but also other European countries, \acs{HAFAS}, short for \textit{\acl{HAFAS}} (HaCon Timetable-Information-System), is a popular system for retrieving timetable information as well as intermodal journey planning. It is developed by a subsidiary of Siemens and available since the late 1980s, and with that one of the first tools on the market, promising an algorithm that performs route calculation in less than six seconds. Each public transport company gets its own \ac{HAFAS} deployment, which all share common terminology and endpoints, but can however use custom configurations and feature availability. While the endpoints are usually not openly documented by the respective companies, they do not restrict access using \ac{API} tokens or similar measures and are therefore easily accessible \autocites{computerwoche-1988}{redmann-2023b}.

For this thesis, the  \ac{HAFAS} endpoint by \ac{VOR}, the transport authority of the eastern regions of Austria, was used for collecting the necessary data. While the \ac{API} itself returns data in the  \ac{HAFAS} Raw Data Format, the JavaScript library \textit{hafas-client} converts this into the much easier-to-work-with \ac{FPTF}, which is inspired by the widely used \ac{GTFS} format and uses \ac{JSON} as its serialization format \autocite{redmann-2023a}.

The library provides a multitude of endpoints, both for station and trip data. Aside from fetching information about a stop or station, the \texttt{nearby} endpoint for example can retrieve all locations that are within a given walking distance in minutes, which is especially useful for building consumer-facing public transport applications. It also includes endpoints for journey planning between two given locations. Similar to \texttt{nearby}, the \texttt{reachableFrom} endpoint returns all stations that are reachable from another station, but instead of walking distance public transport journeys are calculated. Another endpoint that provides opportunities for interesting applications is \texttt{radar}, which returns the locations of all vehicles within a given radius. While location data is not available for all vehicles, this can certainly be used as a foundation for appealing visualizations. The two most important endpoints for this thesis however are \texttt{arrivals} and \texttt{departures}, which provide accurate real-time data for one or more given station(s). In addition to planned and actual arrival/departure times, they also calculate the difference, providing the consumer with a ready-to-use delay value in seconds \autocites{redmann-2023a}{redmann-2023c}.

\subsubsection{\acf{GTFS}}

\ac{GTFS} is a format specification initially developed by Google which later---after its widespread usage in many non-Google systems---has been renamed from \textit{Google Transit Feed Specification} to \textit{General Transit Feed Specification}. It allows public transport agencies to offer their data in a unified format, understood by various consumer applications, ranging from visualization tools to trip-planning mobile applications. The core of \ac{GTFS} is a static feed of text files that contain the whole schedule and additional information about the network. The \ac{GTFS} reference document specifies five required files: \texttt{agency.txt}, \texttt{stops.txt}, \texttt{routes.txt}, \texttt{trips.txt} and \texttt{stop\_times.txt}. One can see that the structure of the data is similar to \ac{HAFAS}. Routes are a group of trips that are offered to passengers as a line or service. Arrival and departure data for those trips are then collected in \texttt{stop\_times.txt} \autocite{gtfs-2022-schedule}.

In contrast to \ac{HAFAS} however, these static files do not provide any historical or real-time data, but instead the planned schedule at the time of publishing. For real-time data, there exists a special extension to \ac{GTFS} called \ac{GTFS} Realtime. As a data format, \ac{GTFS} Realtime uses Protocol Buffers, a language-agnostic serialization format developed by Google. As real-time updates naturally come with a significantly higher level of detail than static text files, the provided data structures are entirely different. Instead of static text files, messages such as \texttt{TripUpdate}, \texttt{VehiclePosition} or \texttt{StopTimeEvent} are exchanged between the feed provider and connected clients \autocite{gtfs-2022-realtime}. 

Since 2017, the city of Vienna provides \ac{GTFS} schedule data as part of the \textit{Open Government Data} initiative. With its real-time extension, \ac{GTFS} would have been able to provide all the necessary data for this thesis, especially planned and actual departure data. However, as the \textit{hafas-client} library uses the \ac{JSON}-based \ac{FPTF} format instead of more complicated Protocol Buffers, \ac{HAFAS} was chosen as a primary data source instead of \ac{GTFS}.