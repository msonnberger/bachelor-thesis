Um die Fahrgastzahlen und die Kundenzufriedenheit in öffentlichen Verkehrsnetzen zu erhöhen, müssen Verkehrsbetriebe verschiedene Maßnahmen ergreifen, um eine attraktive Alternative zu anderen Verkehrsmitteln zu bieten. Neben dem Angebot eines umfangreichen Netzes und der Verbesserung der Infrastruktur ist eine dieser Maßnahmen die Verringerung von Verspätungen und die Bereitstellung zuverlässiger Verbindungen. Um Verspätungen effizient zu reduzieren, ist es wünschenswert, Hotspots im Netz zu erkennen, an denen die meisten Verspätungen auftreten. Ziel dieser Arbeit ist es, Abfahrtsdaten zu analysieren, um potenzielle Hotspots zu finden und zu untersuchen, ob sich diese Hotspots an Umsteigeknoten und Endstationen befinden. Zu diesem Zweck wurden über einen Zeitraum von 30 Tagen Abfahrtsdaten im Wiener Nahverkehrsnetz erhoben. Nach einem Überblick über frühere Untersuchungen in anderen Städten werden die Ergebnisse der Analyse und mögliche Gründe für die Ergebnisse in dieser Arbeit diskutiert. Zusätzlich werden dem Leser Visualisierungen der gesammelten Verspätungsdaten präsentiert, um die Ergebnisse in einer leicht verständlichen Form darzustellen.