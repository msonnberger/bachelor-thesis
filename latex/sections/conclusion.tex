\section{Conclusion}
\label{sec:conclusion}

This section will conclude and summarize the results by highlighting key takeaways and putting them into the context of the scope and goals of this thesis. Finally, aspects that were not in the scope of this thesis and are subject to further research will be mentioned.

The aim of this thesis was to analyze the public transport network of Vienna for on-time performance and detect potential delay hotspots where punctuality rates are particularly low. Public transport is one of the key factors of a functioning livable city and good on-time performance is of high importance in order to ensure high usage and passenger satisfaction. In order to conduct this analysis, a custom data collection script was written which persisted 13.7 million departure records over a period of 30 days. Next, the collected data was cleaned up and filtered for incomplete or irrelevant results. For the following analysis, the data was either further filtered in order to highlight certain aspects, or it was aggregated and summarized as a whole.

Evaluating the data showed that the network offers a very good general on-time performance. While metro lines usually had very few delays due to their separation from other traffic, bus lines had both the highest and lowest punctuality rates, depending on their specific route. The initial assumption was that delays occur more often at busy interchange stations due to lots of passengers getting on and off the vehicles. This assumption did not hold in the general case, as the lowest punctuality rates were found at regular bus stations. On certain lines, however, this effect did become visible, like on tram line 71, which was analyzed in more detail in \cref{sec:line-71-analysis}. Other expected delay hotspots were terminal stations since those provide the opportunity for previously collected delays to accumulate. An interesting observation regarding this was that the terminal stops themselves surprisingly showed the best punctuality rates and even negative mean delays. However, the last 3--5 stops before them had in fact relatively low punctuality rates, suggesting that previous delays do accumulate but are however accounted for by including a leeway in the schedule. Lastly, there were some stations and even parts of whole lines that showed substantially lower punctuality rates than others. In contrast to the aforementioned cases, these however did not show any apparent systemic reason for the bad on-time performances.

In addition to the analysis done in \cref{sec:analysis}, visualizations of the delay data were presented in \cref{sec:visualization}. The aim of those visualizations was to provide an intuitive understanding of the data. The first visualization was a city map with stations plotted at their locations. Each station was colored on a gradient ranging from green to red which represents its punctuality rate. With this, the already known results from \cref{sec:analysis} were visualized as well as new delay hotspots discovered. Next, a line graph was presented which showed the number of departures and the punctuality rate per hour of the day, separated for weekdays and weekends. This showed the expected result that on-time performance drops significantly for morning and afternoon rush hour traffic during weekdays.

The presented statistics and graphics showed that there is definitely room for improvement in Vienna's public transport network, albeit the overall results suggest that there are already various measures put into practice to avoid long delays and with that unsatisfied passengers.

\subsection{Future Work}
\label{sec:future}

There are many opportunities for further analysis of departure and in particular delay data, both in other cities and in Vienna. While the scope of this thesis was limited to departure delays, arrival delays could be analyzed and used in calculations combining arrival and departure times. For example, dwell times---another important measure in public transport analysis---could be calculated and analyzed. With dwell times, stations that actually cause delays could be detected better than when using only departure delays since one can see whether a delay was just carried on from the previous station (indicated by a normal, short dwell time) or a delay emerged at the station in question (indicated by a higher than expected dwell time). Of course, delays can also occur between stations, where this method would not be viable.

As mentioned in previous sections, another important indicator of on-time performance is the intervals between departures. When there are a lot of delays but consistent intervals, passengers are still satisfied if intervals are sufficiently short \autocite[378]{van-oort-2015}. It could be studied how intervals differ from their schedule in order to detect possible delay hotspots. In combination with passenger surveys which could determine acceptable interval deviations, identified hotspots can be narrowed down to a few high-priority locations, where shorter intervals would result in the highest increase in customer satisfaction. Additionally, passenger volume should be taken into consideration. After all, very short intervals are only economically viable if they meet passenger demand.

In addition to the visualizations presented in this thesis, interactive visualizations could be explored in order to provide the reader with even more relevant data and a more engaging experience than a static graph or image. With location data, a live map showing the position of vehicles in real-time would be an interesting interactive application. Even without location data, one could in theory calculate vehicle locations by multiplying the expected minutes or seconds until arrival by the average speed of the vehicle. Naturally, this method can only provide estimations, combined with smooth interpolation and animations this could however result in a useful, visually appealing visualization.

In order to improve the network and the situation for passengers, it is important that after the on-time performance has been studied, concrete measures are implemented that try to resolve identified problems. This can be done by both the transit company which can for example improve vehicles, driver behavior or station infrastructure and also by the city itself and its traffic planning, for example by reducing car lanes in order to avoid delays caused by traffic blocking buses or trams. 
