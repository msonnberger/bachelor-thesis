\section{Introduction}

Good public transportation is one of the most important factors for quality of life in any big city. A study by the European Union found that after frequency, reliability is the second biggest contributing factor to passenger satisfaction with public transport \autocite[66]{eu-2020}. With the European Union's commitment to achieving 55\% less greenhouse gas emission in comparison to 1990, it is of high importance to increase both usage of and satisfaction with public transport. In Vienna, Austria both of these numbers are already at very high levels with 55\% of people using the public transport network on a typical day and 95\% average passenger satisfaction, the best value among capital cities in the EU \autocite[61]{eu-2020}. To be able to reduce delays and thus increase reliability, one has to first identify areas where delays occur most often and to the largest extent. After identifying these areas, possible reasons for occurring delays have to be explored, considering both internal and external reasons \autocite[372]{van-oort-2015}. Only after both of these steps have been completed, effective measures can be researched and implemented.

The aim of this thesis is to identify said problem areas by collecting and analyzing departure data from the transport agency that operates metros, trams and buses in Vienna. Since raw numbers alone often do not provide an intuitive understanding of large data sets, different visualizations will be presented that aim to give the reader an easy-to-understand overview of potential delay hotspots. In particular, the hypothesis that delays occur more often at transfer hubs and terminal stations will be investigated. At transfer hubs, passenger volume is the highest which can lead to increased dwell times and thus delays. At terminal stations on the other hand, delays that occurred at the beginning of a trip can accumulate and therefore form delay hotspots.

The thesis is structured as follows. \Cref{section:related-work} presents previous research and different formats of public transport data that are available in Vienna. \Cref{sec:implementation} first explains some relevant information about the network itself and continues to describe the data collection process in detail. For this process, a custom script was written that persisted the necessary data over a duration of 30 days. Next, the results of the conducted data analysis are presented, highlighting different aspects in varying levels of detail, starting with network-wide analysis and continuing with a closer look at single lines and stations. Additionally, visualizations that provide new views on the data set are presented, showing all stations on a map and coloring them by their punctuality rate in addition to plotting the punctuality rate by time of day. The results and possible reasons causing those results are discussed in \cref{sec:discussion}, mainly focusing on reasons like increased traffic or scheduling effects. Finally, this thesis concludes by summarizing the results found in previous sections. Furthermore, an outlook on possible further research and applications of public transport data is provided.