\subsection{Future Work}
\label{sec:future}

There are many opportunities for further analysis of departure and in particular delay data, both in other cities and in Vienna. While the scope of this thesis was limited to departure delays, arrival delays could be analyzed and used in calculations combining arrival and departure times. For example, dwell times---another important measure in public transport analysis---could be calculated and analyzed. With dwell times, stations that actually cause delays could be detected better than when using only departure delays since one can see whether a delay was just carried on from the previous station (indicated by a normal, short dwell time) or a delay emerged at the station in question (indicated by a higher than expected dwell time). Of course, delays can also occur between stations, where this method would not be viable.

As mentioned in previous sections, another important indicator of on-time performance is the intervals between departures. When there are a lot of delays but consistent intervals, passengers are still satisfied if intervals are sufficiently short \autocite[378]{van-oort-2015}. It could be studied how intervals differ from their schedule in order to detect possible delay hotspots. In combination with passenger surveys which could determine acceptable interval deviations, identified hotspots can be narrowed down to a few high-priority locations, where shorter intervals would result in the highest increase in customer satisfaction. Additionally, passenger volume should be taken into consideration. After all, very short intervals are only economically viable if they meet passenger demand.

In addition to the visualizations presented in this thesis, interactive visualizations could be explored in order to provide the reader with even more relevant data and a more engaging experience than a static graph or image. With location data, a live map showing the position of vehicles in real-time would be an interesting interactive application. Even without location data, one could in theory calculate vehicle locations by multiplying the expected minutes or seconds until arrival by the average speed of the vehicle. Naturally, this method can only provide estimations, combined with smooth interpolation and animations this could however result in a useful, visually appealing visualization.

In order to improve the network and the situation for passengers, it is important that after the on-time performance has been studied, concrete measures are implemented that try to resolve identified problems. This can be done by both the transit company which can for example improve vehicles, driver behavior or station infrastructure and also by the city itself and its traffic planning, for example by reducing car lanes in order to avoid delays caused by traffic blocking buses or trams. 